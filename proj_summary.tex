\documentclass[preprint]{aastex}
\usepackage[top=1in, bottom=1in, left=1in, right=1in]{geometry}
\usepackage{amsmath}
\usepackage{graphicx}
\usepackage{mdwlist}
\usepackage{natbib}
\usepackage{natbibspacing}
\setlength{\bibspacing}{0pt}
\setlength{\parskip}{0pt}
\setlength{\parsep}{0pt}
\setlength{\headsep}{0pt}
\setlength{\topskip}{0pt}
\setlength{\topmargin}{0pt}
\setlength{\topsep}{0pt}
\setlength{\partopsep}{0pt}
\setlength{\footnotesep}{8pt}
\pagestyle{empty}

\def\HI{{H{\small I }}}
\def\HII{{H{\small II }}}

%Project Summary. (1 page maximum) Required elements include an overview of the
%proposed program, and separate entries addressing the intellectual merit and
%broader impacts. The summary should be written in the third person, informative
%to those working in the same or related field(s), and understandable to a
%scientifically or technically literate reader.

\begin{document}
\pagestyle{empty}

\title{HERA: Illuminating Our Early Universe}

This proposal supports the Hydrogen Epoch of Reionization Array (HERA), an experiment optimized to measure 21 cm emission from the primordial intergalactic medium (IGM) throughout cosmic reionization (z=6-12), with support for exploring earlier stages of our Cosmic Dawn (z~30).  During these epochs, early stars and black holes heat and ionize the IGM, introducing fluctuations in 21 cm emission.  HERA characterizes the evolution of the 21 cm power spectrum in detail to constrain the timing and morphology of reionization, the properties of the first galaxies, the evolution of large-scale structure, and the early sources of heating in the IGM.

Following the success of first-generation instruments led by HERA team members (PAPER, MWA, MITEoR, and EDGES), this proposal supports the construction of a 350-element interferometer in South Africa consisting of 14-m parabolic dishes observing from 50 to 250 MHz.  Recent breakthroughs have enabled current instruments (notably PAPER) to control systematics from bright foregrounds, with limits that begin to constrain heating in the early universe.  These instruments are now approaching their sensitivity limits.  HERA brings both the sensitivity and the precision to deliver its primary science on the basis of proven foreground filtering techniques, while developing new subtraction techniques to unlock new capabilities.  The result will be a major step toward realizing the widely recognized scientific potential of 21 cm cosmology.

\section*{Intellectual Merit}

HERA's intellectual merit is to transform our understanding of the first stars, galaxies, and black holes, and their role in driving reionization at the end of our Cosmic Dawn.  Exploring our Cosmic Dawn is one of the three science priorities highlighted by the New Worlds New Horizons 2010 decadal survey, and it is the target of ambitious observing efforts across the electromagnetic spectrum.  HERA brings breakthrough capabilities by using 21 cm emission to link the evolution of ionization in the IGM with the underlying sources driving it. With a new telescope optimized for 3D power-spectral measurements and support for modeling efforts, this proposal delivers the observations and the theoretical framework for constraining the astrophysics of reionization.

HERA's science and data products will have high impact beyond its primary science.  HERA's measurements improve CMB constraints on cosmological parameters and on the sum of neutrino masses by removing the optical depth degeneracy.  HERA extends to lower frequencies with support from the Moore Foundation to find sources of IGM heating prior to reionization.  HERA's public data provide opportunities for cross-correlation studies with other high-redshift probes to characterize the interaction between galaxies and their ionization environment.  Finally, the HERA instrument provides a platform for searching for transient auroral emission from exoplanets and for performing triggered follow-up of fast radio bursts.

\section*{Broader Impact}

HERA's impact is broadened by the CAMPARE-HERA Astronomy Minority Partnership (CHAMP) summer research program, which addresses the NSF's goal of increasing the number of students in underrepresented minority (URM) groups contributing to STEM research.  CHAMP provides an annual 10-week summer research experience shared by US URM undergraduates and South African graduate students. A cohort-building workshop at the beginning provides scientific background and programming skills, and facilitates cultural interaction between these two groups, each underrepresented in science in their own country. Pairs of students then support each other as they work at HERA institutions and present their work at an end-of-summer research symposium.

CHAMP builds on the successful framework of the California Arizona Minority Partnership for Astronomy Research and Education (CAMPARE) program, with its recruiting network of 23 predominantly Hispanic Serving Institutions. With an average of 5 PhDs awarded to URM students in astronomy each year, CHAMP's support of 6-8 students per year can significantly impact the diversity of students advancing in the field of astronomy.


\end{document}
