\documentclass[preprint]{aastex}
\usepackage[top=1in, bottom=1in, left=1in, right=1in]{geometry}
\usepackage{amsmath}
\usepackage{graphicx}
\usepackage{mdwlist}
\usepackage{natbib}
\usepackage{natbibspacing}
\setlength{\bibspacing}{0pt}
\setlength{\parskip}{0pt}
\setlength{\parsep}{0pt}
\setlength{\headsep}{0pt}
\setlength{\topskip}{0pt}
\setlength{\topmargin}{0pt}
%\setlength{\topsep}{0pt}
\setlength{\partopsep}{0pt}
\setlength{\footnotesep}{8pt}
\pagestyle{empty}

\def\HI{{H{\small I }}}
\def\HII{{H{\small II }}}

%Project Summary. (1 page = 4600 characters maximum) Required elements include an overview of the
%proposed program, and separate entries addressing the intellectual merit and
%broader impacts. The summary should be written in the third person, informative
%to those working in the same or related field(s), and understandable to a
%scientifically or technically literate reader.

\begin{document}
\pagestyle{empty}

%Overview: Insert a self-contained description of the activity that would
%result if the proposal were funded and include a statement of objectives and
%methods to be employed. 

\title{HERA: Illuminating Our Early Universe}

We propose to build the Hydrogen Epoch of Reionization Array (HERA), an optimized experiment for studying redshifted 21 cm emission from HI in the primordial intergalactic medium (IGM) throughout cosmic reionization. During this epoch, the first stars and black holes heat and reionize the IGM, introducing fluctuations in the brightness temperature of 21 cm emission. By characterizing the evolution of the 21 cm power spectrum and perhaps by directly imaging reionization, HERA constrains the timing of reionization, the properties of the first galaxies, the evolution of large-scale structure, and the sources of heating in the early IGM.

HERA consists of 352 14-m parabolic dishes operating from 50 to 250 MHz at the South African SKA site. HERA's design incorporates proven foreground avoidance techniques used by PAPER to set the first constraints on early IGM heating and powerful foreground subtraction techniques from the MWA.  With the sensitivity and precision to access the 21 cm signal from the Dark Ages through reionization, HERA opens new windows that will transform our understanding of the early universe.

\section*{Intellectual Merit}
 
Exploring our Cosmic Dawn is one of the three science priorities highlighted by the Astronomy and Astrophysics Decadal Survey (New Worlds, New Horizons, 2011).  Currently, our understanding of cosmic reionization remains rudimentary. When did it occur and over what timescale?  What objects dominated the radiation field? How were they distributed? How did feedback mechanisms in the first galaxies affect these populations? New measurements are needed to advance our theoretical understanding.

The evolution of the 21 cm signal depends on the expansion of the universe, the ignition of the first galaxies, the formation of first massive black holes, and myriad other effects.  This rich history makes HERA measurements uniquely capable of measuring the evolution of the cosmic ionization field, the nature of the first ionizing sources, and the growth of structures in the cosmic web.  It complements other probes high-redshift probes to paint a comprehensive picture of reionization and address measurement degeneracies in fundamental cosmological parameters.  HERA advances the interests of the broader astronomical community by providing data cubes to cross-correlate with, e.g., JWST, CMB maps, and CO intensity mapping, and by serving as a platform for other low frequency science, including fast radio bursts and exoplanetary auroral bursts.  HERA is poised to transform our scientific understanding of the early universe, making it a major step toward unlocking the widely recognized scientific potential of 21 cm cosmology.

\section*{Broader Impact}

HERA's broader impact plan aims to increase participation of underrepresented minority (URM) groups in STEM research and to disseminate HERA science into classrooms. HERA works with the California Arizona Minority Partnership for Astronomy Research and Education (CAMPARE) program and its recruiting network of 22 Hispanic-serving institutions to form the CAMPARE-HERA Astronomy Minority Partnership (CHAMP). CHAMP has two elements: an annual 10-week summer research program involving CAMPARE undergraduates and graduate students from South Africa working on HERA science, and training workshops for school teachers involving CHAMP undergraduates.

The CHAMP summer program begins with an initial cohort-building workshop to provide science background and programming skills. Pairs of CHAMP undergraduates and South African students (each underrepresented in science in their own country) move to HERA institutions, working together for mutual support and continued cultural exchange. At an end-of-summer symposium, participants present their research to an audience of mentors and family members. CHAMP delivers professional development and educational resources for high school classrooms in partnership with the Center for Excellence in Mathematics and Science Teaching (CEMaST) at Cal Poly and UC Berkeley's Multiverse group.  These efforts culminate in visits by CHAMP scholars to science classrooms in their home communities and teacher workshops hosted by CEMaST.

\end{document}
